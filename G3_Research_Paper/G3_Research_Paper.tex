\documentclass{acm_proc_article-sp}
\usepackage[utf8]{inputenc}

\renewcommand{\paragraph}[1]{\vskip 6pt\noindent\textbf{#1 }}
\usepackage{hyperref}
\usepackage{graphicx}
\usepackage{url}

\providecommand{\tightlist}{%
  \setlength{\itemsep}{0pt}\setlength{\parskip}{0pt}}

\title{Short Paper}


% Add imagehandling
\usepackage{graphicx}
% Redefine \includegraphics so that, unless explicit options are
% given, the image width will not exceed the width of the page.
% Images get their normal width if they fit onto the page, but
% are scaled down if they would overflow the margins.
\makeatletter
\def\ScaleIfNeeded{%
  \ifdim\Gin@nat@width>\linewidth
    \linewidth
  \else
    \Gin@nat@width
  \fi
}
\makeatother
\let\Oldincludegraphics\includegraphics
{%
 \catcode`\@=11\relax%
 \gdef\includegraphics{\@ifnextchar[{\Oldincludegraphics}{\Oldincludegraphics[width=\ScaleIfNeeded]}}%
}%

\numberofauthors{2}
\author{
\alignauthor Darryl Kwok \\
        \affaddr{Singapore Management University}\\
       \email{}
\and \alignauthor Megan Sim \\
        \affaddr{Singapore Management University}\\
       \email{}
\and \alignauthor Xu Pengtai \\
        \affaddr{National University of Singapore}\\
       \email{}
\and }

\date{}

%Remove copyright shit
\permission{}
\conferenceinfo{} {}
\CopyrightYear{}
\crdata{}

% Pandoc syntax highlighting

% Pandoc citation processing


\begin{document}
\maketitle

\begin{abstract}
With Geographical Information being readily available everywhere, users
now have a diverse range of data to perform Geospatial Analysis. The
steps taken to prepare the data for in-depth analysis usually differ
across datasets. The subsequent actions executed after data
preprocessing are strikingly similar across Geospatial projects that
falls under the same category i.e.~Geographic Segmentation. To meet the
rising demand for quick but accurate generation of the project results,
we designed and developed a Geographical Segmentation Application that
utilises different clustering algorithms such as, Hierarchical
Clustering, Clustgeo and SKATER.
\end{abstract}

\hypertarget{introduction}{%
\section{Introduction}\label{introduction}}

Comprehensive plans are required to set goals and guidelines for future
growth and development. These plans are required to improve the welfare
of the people and induce the creation of better social, economic and
physical environments. Government bodies often create untargeted and
uniform policies or strategies that hope to address the majority of the
issues in the country. They often fail to consider the specific needs of
each group that are across the country. With spatial information being
affected by many factors, there could be many different factors why and
how occurrences of events are clustered or segregated in certain
locations. We attempt to create a generalised Geographical Segmentation
tool to identify clusters within any datasets to perform in-depth
analysis. Geographic Segmentation divides and separates a target market
into different segments by using geographical location, to better serve
and target each segment specifically. This is often done based on
geographic information and also various other factors such as climate,
cultural preferences, populations and more. If government policies are
untargeted and irrelevant, it will be costly and damaging. Geographical
segmentation is an effective approach for government bodies to identify
the specific needs of each segment to better serve them.

\hypertarget{motivation-and-objectives}{%
\section{Motivation and Objectives}\label{motivation-and-objectives}}

Our research and development were spurred by the increasing need and
demand for quick but accurate execution of geographic segmentation. It
aims to provide the users with the ability to perform parameter tuning
and the flexibility of uploading different datasets to get a quick and
in-depth analysis of the different segments of their data. More
specifically, we attempted to create an application that supports the
following objectives: 1. To be able to upload varying types of datasets
that include geographical and non-geographical data. 2. To create an
interactive visualization that encompasses the different segments,
utilising a dataset of their choice. 3. To provide the users with the
ability to choose an algorithm of their choice from a pool of frequently
used algorithms. 4. To provide the users with the ability to tune and
define the parameters that are required to generate an accurate and
in-depth analysis of 5. their datasets. 5. To reduce the bottleneck
caused by the repetitive acts of generating results from similar
geographical segmentation projects.

\hypertarget{literature-review}{%
\section{Literature Review}\label{literature-review}}

We chose the study on Spanish Employment, Normative versus analytical
regionalisation procedures (Juan Carlos et al, 2004) as our reference
literature because we felt that the approaches taken were the most
similar to our project. One method of study that stood out to us was the
Two-Stage strategy method. In the first stage, the conventional
clustering method was used. Hierarchical, Partitional and k-means
clustering can be chosen as a method of choice for the first stage. At
the second stage, cluster revision in terms of geographical contiguity
will be executed. This is to ensure the homogeneity of regions that were
generated in the first stage. Additionally, it is to obtain evidence of
spatial dependence among the elements. However, it was also found that
the number of groups actually depended on the degree of spatial
dependence rather than the researcher's criteria of study.

In the study, Normative and Analytical Regionalisation were 2 types of
regionalisation used. Although normative regionalisation is a good
method to use, it might not always be appropriate due to some underlying
factors that we might not be able to discover. This will definitely
affect the credibility of the results. On the other hand, analytical
regionalisation takes into account functional zones, geographical
contiguity, equality as well as the interaction between the regions. We
feel that with the consideration of a wide range of factors, analytical
regionalisation is a far more superior approach than normative
regionalisation.

\hypertarget{design-framework}{%
\section{Design Framework}\label{design-framework}}

We designed our data visualization according to Schneiderman's mantra
{[}\textbf{fenner2012a?}{]}: zoom and filter, details on demand. The
types of views available consist of maps and charts, which
correspondingly deal with geospatial and aspatial visualisations.

We have a sequential flow in terms of user journey with the application.
The overall sequence includes: data upload, exploratory data analysis
and in-depth clustering analysis. Multiple clustering analysis methods
have been made available to the user, which covers the main clustering
methods used in both academia and industry. These methods include
hierarchical clustering, clusterGeo and SKATER.

We have a consistent design for each section of the dashboard. For each
page, we have a control panel on the left, which allows users to select
analysis methods according to their own needs. On the right, we show the
detailed results of the analysis in intuitive visualizations.

\begin{figure}
\centering
\includegraphics{}
\caption{A Sample Page for Layout Demonstration}
\end{figure}

\hypertarget{demonstration}{%
\section{Demonstration}\label{demonstration}}

We allow users to upload their own datasets for clustering analysis
(Figure 2).

\begin{figure}
\centering
\includegraphics{}
\caption{Data Upload Page}
\end{figure}

After data uploading, users are able to conduct exploratory data
analysis to decide the clustering method suitable for their case (Figure
3).

\begin{figure}
\centering
\includegraphics{}
\caption{Exploratory Data Analysis Page}
\end{figure}

Users then go to the most suitable clustering analysis method to conduct
detailed analysis in order to understand the clustering pattern of their
geospatial dataset (Figure 4).

\begin{figure}
\centering
\includegraphics{}
\caption{ClustGeo Clustering Analysis Page}
\end{figure}

\hypertarget{discussion}{%
\section{Discussion}\label{discussion}}

With our regionalisation \& geographical segmentation tool, users have a
hassle-free way of obtaining fast visualisations with tweakable
parameters and algorithms. In addition, users who might be new to
spatial analysis get to work with a set of algorithms that are commonly
used in the field, as well as read the explanations on how said
algorithms work and a breakdown of the parameters (such as different
proximity methods).

To use our tool, we require users do a preliminary clean-up of the data
- while we have pre-processing functions to address things like the
coordinate system and missing values, responsibility is on the user to
ensure that their geo spatial and aspatial data are in the appropriate
formats (which are widely used within the field of spatial analysis), as
well as ensure that both datasets have similar features to `join' on.
This serves as a good practice for users to carry with them, not just
for using this tool but also for future geographical analyses they might
explore, where data cleaning and manipulation serves as a good
foundation to build their analyses upon.

Our tool is simple by design: it is meant to set a foundational view or
serve as a preliminary visualisation for analyses requiring geographical
segmentation. However, users who want greater levels of customisation or
deeper insights might want to turn to other tools on the market, or
attempt such regionalisation analysis themselves. Of course, there are
features we would have liked to add to our current tool to make it more
usable for all types of users, which is detailed in our next section.

\hypertarget{future-work}{%
\section{Future Work}\label{future-work}}

While we have attained the primary objective of creating our
regionalisation tool, there are a few improvements that we would have
liked to incorporate given more time and resources. These improvements
are mostly centered on improving user experience by making the tool more
reactive and intuitive, such as: Less stringent data requirements: our
data input requires the aspatial and geospatial data to be in .csv and
.shp format respectively, and to join them, the identifying feature to
join on has to be of the same name. This puts responsibility on the user
to clean up/manipulate their data in order to use our tool, which might
dissuade users from trying the tool, or might even make the tool
unusable for those who have issues with data cleaning + manipulation.
Ideally, an improved version of our tool would be able to take in a
variety of data (in typical formats). Additional pre-processing
features: our preprocessing is the default preprocessing for data meant
for geographical segmentation. This might fulfil the basic requirements
of the user, but they might want to manipulate specific variables or
omit certain sections of the pre-processing. Auto-detect \& suggest
creation of new variables: In our test dataset, we have preemptively
created a new set of variables for gauging the penetration rate (for
each ICT measure), which gives a more accurate sensing of distribution
over a counting-based distribution. However, it would be ideal if the
creation of such new variables is intuitively derived from the dataset:
the user should have some input as to what new variable they want to
derive, and for users who are less sure about which variables are
appropriate for geographical segmentation, the tool should suggest the
recommendation of new variables. Visualisation of segmentation over
time: certain applications of regionalisation, such as a study on the
cases of a particular disease, might want to find out if there was a
change of the regions over time, and if so, what associated variables
could have served as contributing factors. An interactive visualisation
or a .gif of the regions over a period of time would prove to be
beneficial for such cases. Greater customisation: customisation options
such as choosing the palette of the visualisations or being able to
title said visualisations might be important for users who want to use
the visualisations in their work.
\setlength{\parindent}{0in}

\end{document}
